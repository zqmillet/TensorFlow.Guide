% ctex
    \ctexset
	{
		punct              = quanjiao,
		autoindent         = 2,
		contentsname       = 目录,
		listfigurename     = 图目录,
		listtablename      = 表目录,
		figurename         = 图,
		tablename          = 表,
		abstractname       = 摘要,
		appendixname       = 附录,
		appendix/numbering = true,
		section            = {format+= \raggedright}
	}
    \setmainfont{Times Roman}

\usepackage{amssymb, amsmath, bm, mathtools}
	\let\oldvec\vec
	\renewcommand{\vec}[1]{\oldvec{\bm{#1}}}
	\DeclarePairedDelimiter{\abs}{\lvert}{\rvert}
	\def\e{\mathrm{e}}
	\def\dif{\mathrm{d}}

\usepackage{hyperref}
\usepackage{parskip}
	\linespread{1.66}
\usepackage{geometry}
	\geometry
	{
		inner          = 1.1in,
		outer          = 1.1in,
		top            = 1.2in,
		bottom         = 1.2in,
		marginparwidth = 1.3in,
		marginparsep   = 0pt,
		headheight     = 21pt
	}

\usepackage{xcolor}
  	\definecolor{green}{RGB}{0, 127, 0}
	\definecolor{lightgreen}{RGB}{226, 247, 209}
	\colorlet{lightblue}{blue!10}

\usepackage{enumitem}
    \setenumerate[1]{itemsep = 0pt, partopsep = 0pt, parsep = 0pt, topsep = 0pt, label = (\arabic*), leftmargin = \parindent}
    \setitemize[1]  {itemsep = 0pt, partopsep = 0pt, parsep = 0pt, topsep = 0pt, leftmargin = 1em}
    \setitemize[2]  {itemsep = 0pt, partopsep = 0pt, parsep = 0pt, topsep = 0pt, leftmargin = 0pt}
    \setdescription {itemsep = 0pt, partopsep = 0pt, parsep = 0pt, topsep = 0pt}

\usepackage{caption}
	\captionsetup[figure]{
		font = {small, bf},
		labelsep = period,
		textformat = simple,
		skip = 5pt
	}
	\captionsetup[table]{
		font = {small, bf},
		labelsep = period,
		textformat = simple,
		skip = 5pt
	}

\usepackage{titlesec}
	\titleformat*{\section}{\bfseries\fontsize{14pt}{14pt}\selectfont}
	\titlespacing*{\section}{0pt}{10pt}{6pt}
	\titleformat*{\subsection}{\bfseries\fontsize{12pt}{12pt}\selectfont}
	\titlespacing*{\subsection}{0pt}{10pt}{6pt}
	\titleformat*{\subsubsection}{\bfseries\fontsize{12pt}{12pt}\selectfont}
	\titlespacing*{\subsubsection}{0pt}{10pt}{6pt}
	\setcounter{secnumdepth}{4}
	\setcounter{tocdepth}{1}

\usepackage{xkeyval}
\usepackage{etoolbox}
\usepackage{xparse}
\usepackage{multirow}
\usepackage{tabu, longtable}
    \tabulinesep = 3pt
	\AtBeginEnvironment{longtabu}{\small}{}{}
	\AtBeginEnvironment{tabu}{\small}{}{}
	\AtBeginEnvironment{table}{\linespread{1.2}}{}{}
\usepackage[many, minted]{tcolorbox}
	\makeatletter
	\def\inlinebox@true{true}
	\define@key{inlinebox}{frame rule color} {\def\inlinebox@framerulecolor{#1}}
	\define@key{inlinebox}{frame back color} {\def\inlinebox@framebackcolor{#1}}
	\define@key{inlinebox}{frame text color} {\def\inlinebox@frametextcolor{#1}}
	\define@key{inlinebox}{frame rule width} {\def\inlinebox@framerulewidth{#1}}
	\define@key{inlinebox}{banner width}     {\def\inlinebox@bannerwidth{#1}}
	\define@key{inlinebox}{show banner}[true]{\def\inlinebox@showbanner{#1}}
	\define@key{inlinebox}{banner text color}{\def\inlinebox@bannertextcolor{#1}}
	\define@key{inlinebox}{banner back color}{\def\inlinebox@bannerbackcolor{#1}}
	\define@key{inlinebox}{banner text}      {\def\inlinebox@bannertext{#1}}
	\NewDocumentCommand{\inlinebox}{O{} m}{%
		\setkeys{inlinebox}{%
			frame rule color  = black,
			frame back color  = white,
			frame text color  = black,
			frame rule width  = 0.4pt,
			banner width      = 8pt,
			show banner       = false,
			banner text color = white,
			banner back color = black,
			banner text       = BAN,
			#1
		}%
		\tcbox[%
			enhanced,
			tcbox raise base,
			nobeforeafter,
			boxrule           = \inlinebox@framerulewidth,
			top               = -1pt,
			bottom            = -1pt,
			right             = -1pt,
			arc               = 1pt,
			left              = \ifx\inlinebox@showbanner\inlinebox@true\inlinebox@bannerwidth-2pt\else-1pt\fi,
			colframe          = \inlinebox@framerulecolor,
			coltext           = \inlinebox@frametextcolor,
			colback           = \inlinebox@framebackcolor,
			before upper      = {\vphantom{蛤dg}},
			overlay           = {%
				\begin{tcbclipinterior}
				\ifx\inlinebox@showbanner\inlinebox@true
				\fill[\inlinebox@bannerbackcolor] (frame.south west) rectangle node[text = \inlinebox@bannertextcolor, scale = 0.4, font = \sffamily\bfseries, rotate = 90] {\inlinebox@bannertext} ([xshift = \inlinebox@bannerwidth]frame.north west);
				\fi
				\end{tcbclipinterior}%
			}%
		]{#2}%
	}
	\makeatother

	\newcommand{\frameinline}[1]{%
		\inlinebox[%
		]{\mintinline{text}{#1}}}
	\newcommand{\pythoninline}[1]{%
		\inlinebox[%
			frame rule color = green
		]{\mintinline{python}{#1}}}
%

\usepackage{cleveref}
